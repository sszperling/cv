\documentclass[10pt, a4paper, sans]{moderncv}
\usepackage[a4paper, margin=1.5cm, bottom=2.5cm]{geometry}
\usepackage[utf8]{inputenc}
%\usepackage[spanish,es-lcroman]{babel}
\usepackage[english]{babel} % moderncv está buggeado con spanish

\moderncvstyle{seba}
\moderncvcolor{blue}

\name{Sebastián Ariel}{Szperling}
\phone[mobile]{+54 (11) 3511-8184}
\email{zebaszp@gmail.com}
\quote{Soy estudiante de Ciencias de la Computación en la UBA. Me gustan los desafíos y experimentar con nuevas tecnologías, y busco que mi trabajo me sirva para complementar mis estudios.}
\social[linkedin]{seba-sz}
\social[twitter]{ZebaSzp}
\social[github]{ZebaSz}

\begin{document}

\makecvtitle

\section*{Formación}

    \cvitemwithcomment{2014 - actualidad}{Universidad de Buenos Aires, Lic. en Ciencias de la Computación}{Universidad\\ 3er año}

    \cvitemwithcomment{2008 - 2012}{Colegio Nacional de Buenos Aires, Bachiller}{Secundario}

    \cvitemwithcomment{2009}{Cambridge F.C.E. (First Certificate in English)}{Inglés\\ Nota: ”A”}

\section*{Experiencia Laboral}

    \cventry{Feb 2016 - Actualidad}{Analista Desarrollador SSr}{Despegar.com}{Buenos Aires}{}{
        \begin{itemize}
            \item Desarrollador en los equipos de Android
            \begin{itemize}
                \item Desarrollo de dos aplicaciones con más 250k DAU
                \item Organización Agile con historias validadas por equipos de Producto/UX
                \item Diseño de arquitectura basada en patrones como MVP
                \item Estructura de proyecto modular, con lanzamientos frecuentes y estables
                \item Testing de UI automatizado con Robotium y UIAutomator
            \end{itemize}
            \item Capacitación con tecnologías de todo el stack
            \begin{itemize}
                \item Desarrollo back-end en Java usando Spring Framework
                \item Desarrollo front-end en HTML5, CSS3/Sass y JavaScript
                \item Testing y Automatización con Spring RestTemplate, Selenium y TestNG
            \end{itemize}
        \end{itemize}}

\section*{Conocimientos técnicos}

    \cvitem{Lenguajes de programación}{Java (intermedio), C/C++ (intermedio), HTML5 (principiante), CSS3 (principiante), JavaScript (principiante), SQL (principiante), PHP (principiante)}

    \cvitem{Frameworks}{Android SDK, Spring MVC}

    \cvitem{Bibliotecas}{jQuery, JUnit, TestNG}

    \cvitem{Herramientas}{IDEs de JetBrains (IntelliJ, etc.), Maven, Gradle, New Relic, Sass, Selenium, Cassandra, Jenkins, SonarQube, Git, Shell script}

    \cvitem{Otros}{Programación orientada a Objetos y Programación Funcional}

\section*{Idiomas}
    \cvitem{Inglés}{Oral y escrito bilingüe}
    
    \cvitem{Francés}{Oral y escrito básico}

\section*{Actividades y Proyectos personales}
    \cvitem{Proyecto}{Fork del tema para Wordpress WicketPixie: migración de diseño por imágenes a CSS3; actualización a versiones más nuevas del API de Wordpress; implementación de internacionalización; refracción y limpieza de código; instalación de servidor LAMP local para testing; control de versiones – disponible en https://github.com/ZebaSz/wicketpixie}

    \cvitem{Evento}{Participación en la Game Work Jam 2014: programación con Unity3D, UnityScript (JavaScript) y C\#; diseño, organización de proyecto y brainstorming}

\end{document}
